% ==========================
% Packages
% ==========================

\usepackage{chngcntr}
\usepackage{bbm}
\usepackage{xargs}  

% ==========================
% Math Packages
% ==========================

\usepackage{amsmath, amssymb, amsfonts, mathtools}
\usepackage{algorithm}
\usepackage{amsthm}
\usepackage{algpseudocode}
% ==========================
% Graphic Packages
% ==========================
\usepackage{tikz}
\usepackage{tikz-cd}
\usepackage{tikz-3dplot}
\usepackage{xcolor} % For modern color palette
\usepackage{graphicx}
\usepackage{subcaption}
\usepackage{pgfplots}
\usepackage[most]{tcolorbox}
% ==========================
% Tikz Libraries
% =========================
\usetikzlibrary{positioning}
\usetikzlibrary{ arrows.meta}
\usetikzlibrary{shapes.geometric}
\usetikzlibrary{calc}
\usetikzlibrary{arrows}
\usetikzlibrary{shapes, arrows, positioning, fit, calc, backgrounds, chains,decorations.pathreplacing,hobby}
\usetikzlibrary{mindmap,trees}
\usetikzlibrary{automata,positioning}
\usetikzlibrary{decorations.markings}
% ==========================
% Pfg Libraries
% =========================
\usepgfkeyslibrary{ext.pgfkeys-plus}
\usepgfplotslibrary{groupplots}
\pgfplotsset{compat=1.17,colormap/viridis}

% ==========================
% Document Settings
% ==========================

%\addto\captionsenglish{\renewcommand{\bibname}{Bibliography}}
%\setcounter{secnumdepth}{5}
%\setcounter{tocdepth}{1}


% ==========================
% Reference Settings
% ==========================

\newcommand{\secref}[1]{Section \ref{#1}}
\newcommand{\chapref}[1]{section \ref{#1}}
\newcommand{\figref}[1]{Fig \ref{#1}}
\newcommand{\equaref}[1]{(\ref{#1})}
\newcommand{\citepaperof}[2]{\textit{#2} in \cite{#1}}
\newcommand{\citepaperofs}[2]{\textit{#2 et al.} in \cite{#1}}
\newcommand{\algoref}[1]{Algorithm \ref{#1}}
\newcommand{\defref}[1]{Definition \ref{#1}}
\newcommand{\tabref}[1]{Table \ref{#1}}
\newcommand{\ruleref}[1]{Rule \ref{#1}}
\newcommand{\subfigref}[1]{(\subref{#1})} 
\newcommand{\appref}[1]{Appendix \ref{#1}}
\newcommand{\sectionbreak}{\clearpage}


% ==========================
% Theorem Style
% ==========================

%\theoremstyle{definition}
%\newtheorem{definition}{Definition}
%
%\theoremstyle{definition}
%\newtheorem{statecomprule}{SC-Rule}
%
%\theoremstyle{definition}
%\newtheorem{norm}{Norm}

% ==========================
% Colors
% ==========================
\definecolor{viridisgreencolor}{RGB}{32, 162, 57} % Viridis green color
\definecolor{viridisorangecolor}{RGB}{253, 174, 97}
\definecolor{viridisbluecolor}{RGB}{68, 1, 84}
\definecolor{viridisyellowcolor}{RGB}{253, 231, 37}
\definecolor{viridispurplecolor}{RGB}{108, 60, 142}
\definecolor{viridiscyancolor}{RGB}{1, 133, 113}
\definecolor{viridismagentacolor}{RGB}{177, 3, 155}


\definecolor{layer1}{HTML}{4CAF50} % Green for perception layer
\definecolor{layer2}{HTML}{2196F3} % Blue for network layer
\definecolor{layer3}{HTML}{FFC107} % Yellow for computation layer
\definecolor{layer4}{HTML}{F44336} % Red for control layer
\definecolor{layer5}{HTML}{9C27B0} % Purple for application layer
\definecolor{subLayer1}{HTML}{388E3C} % Darker green for sublayer 1
\definecolor{subLayer2}{HTML}{66BB6A} % Lighter green for sublayer 2


% ==========================
% Tikkz styles
% ==========================

\tikzstyle{layer} = [rectangle, rounded corners, text centered, draw=black!50, dashed, inner sep= 15pt]
\tikzstyle{layerTight} = [rectangle, rounded corners, minimum width=6.3cm, minimum height=1cm, text centered, draw=black!50, dashed, inner sep= 5pt]
\tikzstyle{layerGroup} = [rectangle, rounded corners, minimum width=7.3cm, minimum height=1cm, text centered, draw= black, dashed, inner sep= 5pt]
\tikzstyle{subLayer} = [rectangle, rounded corners, minimum width=2.8cm, minimum height=3cm, text centered, draw=black]
\tikzstyle{arrow} = [thick,->,>=stealth]
\tikzstyle{arrowDown} = [thick,<-,>=stealth]
\tikzstyle{doublearrow} = [thick,<->,>=stealth]
\tikzstyle{title} = [font=\color{black}\fontsize{9}{9}\selectfont, opacity=0.5]
\tikzset{arrowstyle/.style={draw=black, fill=layer5!30, single arrow,single arrow,
		single arrow head extend=.4cm,}}
\newcommand{\asymcloud}[2][.1]{%Cloud shape
	\begin{scope}[#2]
		\pgftransformscale{#1}%    
		\pgfpathmoveto{\pgfpoint{261 pt}{115 pt}} 
		\pgfpathcurveto{\pgfqpoint{70 pt}{107 pt}}
		{\pgfqpoint{137 pt}{291 pt}}
		{\pgfqpoint{260 pt}{273 pt}} 
		\pgfpathcurveto{\pgfqpoint{78 pt}{382 pt}}
		{\pgfqpoint{381 pt}{445 pt}}
		{\pgfqpoint{412 pt}{410 pt}}
		\pgfpathcurveto{\pgfqpoint{577 pt}{587 pt}}
		{\pgfqpoint{698 pt}{488 pt}}
		{\pgfqpoint{685 pt}{366 pt}}
		\pgfpathcurveto{\pgfqpoint{840 pt}{192 pt}}
		{\pgfqpoint{610 pt}{157 pt}}
		{\pgfqpoint{610 pt}{157 pt}}
		\pgfpathcurveto{\pgfqpoint{531 pt}{39 pt}}
		{\pgfqpoint{298 pt}{51 pt}}
		{\pgfqpoint{261 pt}{115 pt}}
		\pgfusepath{fill,stroke}         
\end{scope}}    


\newcommand{\defineplane}[7]{%
	% Inputs: right uppermost point (#1, #2, #3), base (#4), height (#5), fill style (#6), vertex names (#7)
	\coordinate (#7-TopRight) at (#1, #2, #3);
	\coordinate (#7-TopLeft) at (#1-#4, #2, #3);
	\coordinate (#7-BottomRight) at (#1, #2-#5, #3);
	\coordinate (#7-BottomLeft) at (#1-#4, #2-#5, #3);
	\coordinate (#7-MiddlePoint) at (#1/4+#1/4-#4/4+#1/4+#1/4-#4/4,         #2/4+#2/4+#2/4-#5/4+#2/4-#5/4,            #3);
	
	% Draw the plane
	\filldraw[#6] (#7-TopRight) -- (#7-TopLeft) -- (#7-BottomLeft) -- (#7-BottomRight) -- cycle;
	\draw[thick] (#7-TopRight) -- (#7-TopLeft) -- (#7-BottomLeft) -- (#7-BottomRight) -- cycle;
	\node[fill=none, circle, inner sep=0pt] (#7-MiddlePoint) at (#7-MiddlePoint) {}; 
	
	% Automatically define the anchor points for the plane (center of each edge)
	\coordinate (#7-west) at ($(#7-TopLeft)!0.5!(#7-BottomLeft)$);   % west anchor at the center of the left edge
	\coordinate (#7-east) at ($(#7-TopRight)!0.5!(#7-BottomRight)$);  % east anchor at the center of the right edge
	\coordinate (#7-north) at ($(#7-TopLeft)!0.5!(#7-TopRight)$);    % north anchor at the center of the top edge
	\coordinate (#7-south) at ($(#7-BottomLeft)!0.5!(#7-BottomRight)$);  % south anchor at the center of the bottom edge
	
}



\tikzdeclarecoordinatesystem{hexaring}{%
	\pgfqkeys{/tikz/cs}{#1}%
	\pgfmathtruncatemacro\hRing{\pgfkeysvalueof{/tikz/cs/ring}}%
	\ifnum\hRing=0 \pgfpointorigin
	\else
	\pgfmathtruncatemacro\hPos{\pgfkeysvalueof{/tikz/cs/pos}}%
	\pgfmathtruncatemacro\hSide{\hPos/\hRing}%
	\pgfmathtruncatemacro\hPosOnSide{mod(\hPos,\hRing)}%
	\pgfpointlineattime{\hPosOnSide/\hRing}
	{\pgfpointpolarxy{30+\hSide*60}{1.73205*\hRing}}
	{\pgfpointpolarxy{90+\hSide*60}{1.73205*\hRing}}%
	\fi}
\tikzset{
	cs/ring/.initial=0, cs/pos/.initial=0, list/.initial={0,...,4},
	hexagon path/.style 2 args={
		insert path={plot[sharp cycle,samples at={0, 60, ..., 359}](\x:1)}},
	hexagon/.style 2 args={
		hexagon ring #1/.try={#2}, hexagon pos #2/.try={#1},
		/utils/TeX/ifnum={#1=0}{}{
			hexagon not ring 0/.try={#1}{#2},
			style/.pgfmath wrap={hexagon side ##1/.try={#1}{#2}}{int(#2/#1)},
			style/.pgfmath wrap={hexagon pos' ##1/.try={#1}{#2}}{int(mod(#2,#1))}}}}
\newcommand*\tikzhexagonofhexagons[1][]{%
	\foreach[/utils/exec=\tikzset{#1}, expand list,
	evaluate={\hexPerRing=int(max(1,\ring*6)-1);}]\ring in{\pgfkeysvalueof{/tikz/list}}
	\foreach[/tikz/cs/ring=\ring]\pos in {0, ..., \hexPerRing}
	\draw[hexagon={\ring}{\pos}, shift={(hexaring cs: pos=\pos)}, hexagon path={\ring}{\pos}];}


% ==========================
% Boxes for cool definitions, examples etc
% ==========================


\tcbset{
	myboxstyle/.style={
		enhanced, breakable,
		sharp corners,
		left skip=8mm,
		attach boxed title to top left={yshift*=-\tcboxedtitleheight/2},
		boxed title style={colframe=#1!40, height=6mm, bean arc, boxrule=1pt},
		colback=#1!10, colframe=#1!30, boxrule=1pt,
		coltitle=black, colbacktitle=#1!10,
		fonttitle=\bfseries,
		underlay boxed title={%
			\node[circle, fill=#1!10, draw=#1!40, inner sep=1pt, minimum size=6mm, line width=1pt, font=\bfseries] (boxnumber) at ([xshift=-6mm]title.west) {\thetcbcounter};
		},
		underlay unbroken={%
			\draw[#1!40, line width=1pt] (title.west)--(boxnumber);
			\draw[#1!40, -|, line width=1pt] (boxnumber)--(boxnumber|-frame.south);
		},
		underlay first={%
			\draw[#1!40, line width=1pt] (title.west)--(boxnumber);
			\draw[#1!40, line width=1pt] (boxnumber)--(boxnumber|-frame.south);
		},
		underlay middle={%
			\draw[#1!40, line width=1pt] (boxnumber|-frame.north)--(boxnumber|-frame.south);
		},
		underlay last={%
			\draw[#1!40, -|, line width=1pt] (boxnumber|-frame.north)--(boxnumber|-frame.south);
		},
	}
}

% Define the boxed environment
\newtcolorbox{definitionbox}[2][]{%
	colback=white, % Background color
	colframe=black, % Border color
	fonttitle=\bfseries, % Title font
	title=\textbf{Definition \arabic{definitionCounter}}:~\textit{#1}, % Title format
	sharp corners, % Box style
}


\newtcolorbox[use counter=definitionCounter]{definitionBox}[2][]{%
	myboxstyle=blue, title=\textbf{Definition}:~\textit{#2}, #1
}

\newtcolorbox[use counter=exampleCounter]{exampleBox}[1][]{%
	myboxstyle=green, title=\textbf{Example}, #1
}


\newtcolorbox[use counter=assumptionCounter]{assumptionBox}[1][]{%
	myboxstyle=red, title=\textbf{Assumption}, #1
}
\newtcolorbox[use counter=propertyCounter]{propertyBox}[1][]{%
	myboxstyle=cyan, title=\textbf{Property}, #1
}


% ==========================
% Counters
% ==========================


\newcounter{exampleCounter}
\newcounter{definitionCounter}
\newcounter{assumptionCounter}
\newcounter{propertyCounter}


% ==========================
% Useful Command to write notes, missing citations etc
% ==========================
\newcommand{\citationRequired}{\textcolor{red}{(CITATION)}}


\usepackage[colorinlistoftodos,prependcaption,textsize=tiny]{todonotes}
\newcommandx{\unsure}[2][1=]{\todo[linecolor=red,backgroundcolor=red!25,bordercolor=red,#1]{#2}}
\newcommandx{\change}[2][1=]{\todo[linecolor=blue,backgroundcolor=blue!25,bordercolor=blue,#1]{#2}}
\newcommandx{\info}[2][1=]{\todo[linecolor=OliveGreen,backgroundcolor=OliveGreen!25,bordercolor=OliveGreen,#1]{#2}}
\newcommandx{\improvement}[2][1=]{\todo[linecolor=Plum,backgroundcolor=Plum!25,bordercolor=Plum,#1]{#2}}
\newcommandx{\thiswillnotshow}[2][1=]{\todo[disable,#1]{#2}}

\newcommand{\tabitem}{~~\llap{\textbullet}~~}


\newenvironment{missingelements}[1][Missing Elements]
{%
	\begin{tcolorbox}[colframe=red, colback=white, boxrule=1pt, arc=0pt]
		\textbf{\color{red}#1} % Title in bold and red
		\begin{itemize}
		}
		{%
		\end{itemize}
	\end{tcolorbox}
}


\newenvironment{ideastoinclude}[1][Possible ideas]
{%
	\begin{tcolorbox}[colframe=blue, colback=white, boxrule=1pt, arc=0pt]
		\textbf{\color{blue}#1} % Title in bold and red
		\begin{itemize}
		}
		{%
		\end{itemize}
	\end{tcolorbox}
}

\newenvironment{whattocheck}[1][What's there to check]
{%
	\begin{tcolorbox}[colframe=orange, colback=white, boxrule=1pt, arc=0pt]
		\textbf{\color{orange}#1} % Title in bold and red
		\begin{itemize}
		}
		{%
		\end{itemize}
	\end{tcolorbox}
}