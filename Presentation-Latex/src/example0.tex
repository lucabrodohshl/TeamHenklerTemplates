\begin{frame}{Options to set with \texttt{\textbackslash thset\{\}} command  }
    \begin{table}[h]
        \centering
        \begin{tabular}{|l|l|}
            \hline
            \textbf{Key} & \textbf{Options} \\ 
            \hline
            uni & HSHL, Fhd \\ \hline
            inner/sectionpage & none, simple, \\&progressbar, progressbarHSHL \\ \hline
            inner/subsectionpage & none, simple, progressbar \\ \hline
            color/block & transparent, fill \\ \hline
            color/background & dark, light \\ \hline
            outer/footlinestyle & plain, slick \\ \hline
            outer/numbering & none, counter, fraction \\ \hline
            outer/progressbar & none, head, frametitle, \\& foot, headstatic, \\&frametitlestatic, footstatic \\ \hline
        \end{tabular}
       
    \end{table}
    \end{frame}
    
    \section[Example0]{Talking about a theorem}
    \begin{frame} 
    \frametitle{There Is No Largst Prime Number} 
    \framesubtitle{The proof uses \textit{reductio ad absurdum}.} 
    \begin{theorem}
    There is no largest prime number. \end{theorem} 
    \begin{enumerate} 
    \item<1-| alert@1> Suppose $p$ were the largest prime number. 
    \item<2-> Let $q$ be the product of the first $p$ numbers. 
    \item<3-> Then $q+1$ is not divisible by any of them. 
    \item<1-> But $q + 1$ is greater than $1$, thus divisible by some prime
    number not in the first $p$ numbers. 
    \end{enumerate}
    \end{frame}
  
    